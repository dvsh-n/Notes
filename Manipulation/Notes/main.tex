\documentclass[12pt]{article}
\usepackage[margin=1in]{geometry}
\usepackage{amsmath}

\title{Notes}
\author{}
\date{}

\begin{document}

\maketitle

\section{PLanning Optimal Grasps}
\subsection{Force Closure grasps}

Force closure grasp refers to the ability of a robotic hand to grasp an object in such a way that it can resist any external force applied to the object. This means that the object will not slip or move regardless of the direction of the applied force.

Mathematically, a grasp achieves force closure if the convex hull of the contact wrenches (forces and torques) includes the origin. This implies that the contact points and the forces they can exert span the entire space of possible forces and torques, allowing the grasp to counteract any disturbance.

Intuitively, imagine holding a ball with your fingers. If you can apply forces in such a way that the ball cannot move or slip out of your hand no matter how it is pushed, you have achieved a force closure grasp.

\subsubsection{Wrenches}

A wrench combines force and torque into a six-dimensional vector. The first three components represent the force, and the last three represent the torque.

Mathematically, a wrench \( \mathbf{w} \) can be represented as:
\[
\mathbf{w} = \begin{bmatrix}
\mathbf{f} \\
\mathbf{\tau}
\end{bmatrix}
\]
where \( \mathbf{f} \) is the force vector and \( \mathbf{\tau} \) is the torque vector.
Wrenches are crucial in robotic grasp analysis as they describe the combined effect of forces and torques applied by a robotic hand. By examining the wrenches at contact points, we can determine if the grasp can resist external disturbances and achieve force closure.

\subsubsection{Friction Cone}

The friction cone is a concept used to describe the range of possible contact forces that can be exerted at a contact point without slipping. It is defined by the friction coefficient \( \mu \) and the normal force \( \mathbf{n} \) at the contact point.

Mathematically, the friction cone can be represented as:
\[
\mathbf{f} = \alpha \mathbf{n} + \beta \mathbf{t}
\]
where \( \mathbf{t} \) is a vector tangent to the contact surface, and \( \alpha \) and \( \beta \) are scalars such that:
\[
\sqrt{\alpha^2 + \beta^2} \leq \mu \|\mathbf{n}\|
\]

This inequality ensures that the contact force \( \mathbf{f} \) lies within the cone defined by the friction coefficient and the normal force, preventing slipping at the contact point.

\subsection{Quality Measures for Grasps}
\[Q = \min_{w \in \text{Bd}(BG)} \|w\|\]
Lets say you have a grasp and you have wrenches in mulitple directions to tets it form using a wrench \(w\). At every different direction of \(w\) the grasp can resist a certain maximum magnitude of \(w\) which is represented by boundary of \(BG\): \(Bd(BG)\) where BG is the set of all wrenches that the grasp can resist and \(G\) is the union of all grasps.
There exists a sphere in that space where the grasp can resist a maximum wrench of magnitude \(Q\) in every direction. Hence \(Q\) is is a measure of robustness of a grasp where a grasp with higher quality can resist larger wrenches in all directions. 

\subsection{Minimizing theaximum finger force}
\subsubsection{Force at the i-th Contact (\(f_i\))}

At a given contact point \(i\), the force \(f_i\) can be expressed as:
\[
f_i = \sum_{j=1}^{m} \alpha_{i,j} f_{i,j}
\]
Where:
\begin{itemize}
    \item \(f_{i,j}\): Basis vectors that define the friction cone.
    \item \(\alpha_{i,j} \geq 0\): Non-negative scalar coefficients for the convex combination.
    \item \(\sum_{j=1}^{m} \alpha_{i,j} \leq 1\): Ensures the force stays within the friction cone.
\end{itemize}

\subsubsection{Torque at the i-th Contact (\(\tau_i\))}

The torque generated by the contact force is given by:
\[
\tau_i = r_i \times f_i
\]
Where:
\begin{itemize}
    \item \(r_i\): Vector from the object's center of mass to the contact point.
    \item \(\times\): Cross-product.
\end{itemize}

Substituting \(f_i\), the torque becomes:
\[
\tau_i = \sum_{j=1}^{m} \alpha_{i,j} (r_i \times f_{i,j})
\]

\subsubsection{Wrench at the i-th Contact (\(w_i\))}

A wrench is a combination of force and torque:
\[
w_i = \begin{bmatrix}
f_i \\
\tau_i
\end{bmatrix}
\]

Substituting the expressions for \(f_i\) and \(\tau_i\), the wrench is:
\[
w_i = \sum_{j=1}^{m} \alpha_{i,j} w_{i,j}
\]
Where \(w_{i,j}\) represents the wrenches corresponding to the \(j\)-th basis vector.

\subsubsection{Set of Wrenches for Each Contact (\(W_i\))}

The set of all possible wrenches from the \(i\)-th contact is:
\[
W_i = \left\{ w_i \mid w_i = \sum_{j=1}^{m} \alpha_{i,j} w_{i,j}, \alpha_{i,j} \geq 0, \sum_{j=1}^{m} \alpha_{i,j} \leq 1 \right\}
\]
This defines the feasible range of wrenches that can be applied at the contact point, bounded by the friction cone.

\subsubsection{Total Wrench on the Object (\(w\))}

The total wrench acting on the object is the sum of the wrenches from all contact points:
\[
w = \sum_{i=1}^{n} w_i
\]
Where \(n\) is the number of contact points.

\subsubsection{Set of All Possible Wrenches (\(BG\))}

The set of all wrenches that the grasp can produce is given by the Minkowski sum of the individual wrench sets \(W_i\):
\[
BG = W_{L^\infty} = W_1 \oplus W_2 \oplus \cdots \oplus W_n
\]

\subsubsection{Convex Hull}
\[
W_{L^\infty} = \text{ConvexHull}\left( \bigoplus_{i=1}^{n} \{w_{i,1}, w_{i,2}, \ldots, w_{i,m}\} \right)
\]
The Convex Hull forms a geometric shape (e.g., a polyhedron in 3D or a polygon in 2D) that represents the full set of possible wrenches the grasp can generate. It is more efficient to calculate than a minkowski sum.

\subsection{Minimizing the total finger force}
\subsubsection{Total Force (\(f\))}

The total force \(f\) acting on the object is the sum of the forces at all contact points:
\[
f = \sum_{i=1}^{n} f_i
\]
Where:
\begin{itemize}
    \item \(f_i\): Force at the \(i\)-th contact point.
\end{itemize}

Each \(f_i\) lies within the friction cone, represented as a convex combination of basis vectors \(f_{i,j}\):
\[
f = \sum_{i=1}^{n} \sum_{j=1}^{m} \alpha_{i,j} f_{i,j}
\]
With:
\begin{itemize}
    \item \(\alpha_{i,j} \geq 0\)
    \item \(\sum_{i=1}^{n} \sum_{j=1}^{m} \alpha_{i,j} \leq 1\): Ensures the forces remain in the friction cone.
\end{itemize}

\subsubsection{Total Wrench (\(w\))}

Similarly, the total wrench acting on the object is:
\[
w = \sum_{i=1}^{n} w_i
\]
Where \(w_i\) is the wrench at the \(i\)-th contact point, and it can be expressed as:
\[
w = \sum_{i=1}^{n} \sum_{j=1}^{m} \alpha_{i,j} w_{i,j}
\]

\subsubsection{Set of All Possible Wrenches (\(W_{L^1}\))}

The set of all wrenches that the grasp can produce, under this criterion, is represented by:
\[
W_{L^1} = \text{ConvexHull}\left( \bigcup_{i=1}^{n} \{w_{i,1}, \ldots, w_{i,m}\} \right)
\]


\end{document}