\documentclass[12pt]{article}
\usepackage[margin=0.5in]{geometry}
\usepackage{amsmath}
\usepackage{amssymb}

\setlength{\parindent}{0pt} % Remove automatic indenting at new paragraphs
\setlength{\parskip}{\baselineskip} % Add space between paragraphs

\title{Homework 7}
\author{Devesh}
\date{\today}

\begin{document}

\maketitle

\section*{Problem 1}
Consider the 3-dimensional, single-input system
\[
\dot{x} = Ax + bu,
\]
where \( A \in \mathbb{R}^{3 \times 3} \) is a diagonal matrix and \( b = \begin{pmatrix} b_1 \\ b_2 \\ b_3 \end{pmatrix} \). Suppose all the diagonal elements of \( A \) are different from each other. Derive a necessary and sufficient condition on \( b \) such that the system is completely controllable.

\textbf{Answer:}

\[
\mathcal{C} = \begin{bmatrix} b & Ab & A^2 b \end{bmatrix} = \begin{bmatrix}
b_1 & \lambda_1 b_1 & \lambda_1^2 b_1 \\
b_2 & \lambda_2 b_2 & \lambda_2^2 b_2 \\
b_3 & \lambda_3 b_3 & \lambda_3^2 b_3 \\
\end{bmatrix}.
\]
\[
\mathcal{C} = \operatorname{diag}(b_1, b_2, b_3) \begin{bmatrix}
1 & \lambda_1 & \lambda_1^2 \\
1 & \lambda_2 & \lambda_2^2 \\
1 & \lambda_3 & \lambda_3^2 \\
\end{bmatrix}.
\]
\[
\det \mathcal{C} = b_1 b_2 b_3 \cdot \det \begin{bmatrix}
1 & \lambda_1 & \lambda_1^2 \\
1 & \lambda_2 & \lambda_2^2 \\
1 & \lambda_3 & \lambda_3^2 \\
\end{bmatrix} = b_1 b_2 b_3 (\lambda_2 - \lambda_1)(\lambda_3 - \lambda_1)(\lambda_3 - \lambda_2).
\]
Since the \( \lambda_i \) are distinct, the determinant is nonzero if and only if \( b_1 b_2 b_3 \ne 0 \). Therefore, the system is completely controllable if and only if all components of \( b \) are nonzero.

\textbf{Necessary and sufficient condition:}
\[
b_i \ne 0 \quad \text{for} \quad i = 1, 2, 3.
\]


\section*{Problem 2}
Consider a single-input, single-output system whose input-output transfer function is
\[
G(s) = \frac{s + 2}{(s + 1)^2}.
\]

\begin{enumerate}
    \item Derive a state-space realization in controllable canonical form. Specify the resulting matrices \( A \), \( b \), and \( c \).
    \item Compute a matrix \( K \in \mathbb{R}^{1 \times 2} \) such that the eigenvalues of the matrix \( A - bK \) are at \( -3 \pm 2j \).
\end{enumerate}

\textbf{Answer:}

\textbf{1:}
\[G(s) = \frac{Y(s)}{U(s)} = \frac{s+2}{(s^2+2s+1)}\]
\[\frac{\hat{Y}(s)}{U(s)} = \frac{1}{s^2+2s+1}\]
\[\ddot{\hat{y}} + 2\dot{\hat{y}} + \hat{y} = u\]
\[\hat{y} = x_1, \dot{\hat{y}} = x_2\]
\[
\begin{bmatrix}
\dot{x}_1 \\
\dot{x}_2
\end{bmatrix}
=
\begin{bmatrix}
0 & 1 \\
-1 & -2
\end{bmatrix}
\begin{bmatrix}
x_1 \\
x_2
\end{bmatrix}
+
\begin{bmatrix}
0 \\    
1
\end{bmatrix}
u
\]
\[Y(s) = (s+2)\hat{Y}(s)\]
\[y = \dot{\hat{y}} + 2\hat{y} = x_2 + 2x_2\]
\[
y = \begin{bmatrix}
2 & 1
\end{bmatrix}
\begin{bmatrix}
x_1 \\
x_2
\end{bmatrix}
\]

\textbf{2:}
\[
A = \begin{bmatrix}
0 & 1 \\
-1 & -2
\end{bmatrix}
\]
\[
K = \begin{bmatrix}
k_1 & k_2
\end{bmatrix}
\]
\[
A-bk = \begin{bmatrix}
0 & 1 \\
-1-k1 & -2-k2
\end{bmatrix}  
\]
\[det(A-bk-\lambda I) = \lambda^2 + k_2\lambda + 2\lambda + k_1 + 1\]
\[(\lambda+3-2j)(\lambda+3+2j) = 0\]
\[\lambda^2-6\lambda+13=0\]
\[k_1=12, k_2=-8\]

\section*{Problem 3}
\section*{Problem 3}

Consider the system
\[
\dot{x} = Ax + bu, \quad y = cx,
\]
where
\[
A = \begin{bmatrix}
1 & 1 & 1 \\
0 & 1 & 0 \\
0 & 1 & 1
\end{bmatrix}, \quad
b = \begin{bmatrix}
0 \\ 0 \\ 1
\end{bmatrix}, \quad
c = \begin{bmatrix}
0 & 0 & 1
\end{bmatrix}.
\]

Compute the rank of \(\mathcal{O}\):
\[
\operatorname{rank}(\mathcal{O}) = 2.
\]

Since \(\operatorname{rank}(\mathcal{O}) < 3\), the system is not observable.

Therefore, the system is neither controllable nor observable. 
\textbf{(c) Find the Kalman decomposition.}

\textbf{Answer:}

The controllability matrix is:
\[
\mathcal{C} = \begin{bmatrix} b & Ab & A^2b \end{bmatrix} = \begin{bmatrix}
0 & 1 & 2 \\
0 & 0 & 0 \\
1 & 1 & 1
\end{bmatrix}.
\]
Its rank is 2, so the controllable subspace has dimension 2.

The observability matrix is:
\[
\mathcal{O} = \begin{bmatrix} c \\ cA \\ cA^2 \end{bmatrix} = \begin{bmatrix}
0 & 0 & 1 \\
0 & 1 & 1 \\
0 & 2 & 1
\end{bmatrix}.
\]
Its rank is 2, so the observable subspace has dimension 2.

$c\bar{o} = \begin{bmatrix} 1 \\ 0 \\ 0 \end{bmatrix}$,
$span(C) = \begin{bmatrix}
    1 & 0 \\   
    0 & 1 \\
    0 & 0 \\  
\end{bmatrix}$

Supplement with $c\bar{o}$ to span(C):

$co = \begin{bmatrix} 0 \\ 1 \\ 0\end{bmatrix}$

Null(O) = $span(\begin{bmatrix} 1 \\ 0 \\ 0\end{bmatrix})$

Supplement with $c\bar{o}$ to null(O):

$\bar{c}\bar{o} = \begin{bmatrix} 1 \\ 0\\ 0\end{bmatrix}$

Supplement everything to Rn with $\bar{c}o$:

$\bar{c}o = \begin{bmatrix} 0 \\ 0 \\ 1\end{bmatrix} $

\[
P = \begin{bmatrix}
    0 & 1 & 0 \\   
    1 & 0 & 0 \\
    0 & 0 & 1 \\
\end{bmatrix}
\]

\[
\bar{A} = P^{-1}AP = \begin{bmatrix} 
1 & 0 & 0 \\
1 & 1 & 1 \\
1 & 0 & 1
\end{bmatrix}
\]

\[
\bar{b} = P^{-1}b = \begin{bmatrix} 0 \\ 0 \\ 1\end{bmatrix}
\]

\[
\bar{c} = cP = \begin{bmatrix} 0 & 0 & 1\end{bmatrix}
\]

\text{Minimal realization:}

\[
\bar{A}_{min} = \begin{bmatrix}
1 & 1 \\
0 & 1
\end{bmatrix}
\]

\[
\bar{b}_{min} = \begin{bmatrix}
1 \\ 1
\end{bmatrix}
\]

\[
\bar{c}_{min} = \begin{bmatrix}
0 & 1
\end{bmatrix}
\]

The minimal realization is given by:
\[
\bar{A}_{min} = \begin{bmatrix}
1 & 1 \\
0 & 1
\end{bmatrix}, \quad
\bar{b}_{min} = \begin{bmatrix}
1 \\ 1
\end{bmatrix}, \quad
\bar{c}_{min} = \begin{bmatrix}
0 & 1
\end{bmatrix}.
\]

The transfer function \( G(s) \) can be found using the formula:
\[
G(s) = \bar{c}_{min} (sI - \bar{A}_{min})^{-1} \bar{b}_{min}.
\]

First, compute \( (sI - \bar{A}_{min}) \):
\[
sI - \bar{A}_{min} = \begin{bmatrix}
s-1 & -1 \\
0 & s-1
\end{bmatrix}.
\]

Next, find the inverse of \( (sI - \bar{A}_{min}) \):
\[
(sI - \bar{A}_{min})^{-1} = \begin{bmatrix}
\frac{1}{s-1} & \frac{1}{(s-1)^2} \\
0 & \frac{1}{s-1}
\end{bmatrix}.
\]

Now, compute the transfer function:
\[
G(s) = \bar{c}_{min} \begin{bmatrix}
\frac{1}{s-1} & \frac{1}{(s-1)^2} \\
0 & \frac{1}{s-1}
\end{bmatrix} \bar{b}_{min} = \begin{bmatrix}
0 & 1
\end{bmatrix} \begin{bmatrix}
\frac{1}{s-1} & \frac{1}{(s-1)^2} \\
0 & \frac{1}{s-1}
\end{bmatrix} \begin{bmatrix}
1 \\ 1
\end{bmatrix}.
\]

Simplifying the multiplication:
\[
G(s) = \begin{bmatrix}
0 & 1
\end{bmatrix} \begin{bmatrix}
\frac{1}{s-1} \\
\frac{1}{s-1}
\end{bmatrix} = \frac{1}{s-1}.
\]

Therefore, the transfer function is:
\[
G(s) = \frac{1}{s-1}.
\]

\section*{Problem 4}

Controllability Matrix \(\mathcal{C}\):

\[
\mathcal{C} = \begin{bmatrix} b & Ab & A^2b \end{bmatrix}
\]

Compute \( Ab \) and \( A^2b \):

\[
Ab = A b = \begin{bmatrix}
-1 & -2.5 & 0.5 \\
2 & 4 & -1 \\
1 & 1.5 & 0.5
\end{bmatrix}
\begin{bmatrix}
1 \\ 1 \\ 3
\end{bmatrix}
= \begin{bmatrix}
-2 \\
3 \\
4
\end{bmatrix}
\]

\[
A^2b = A (Ab) = A \begin{bmatrix}
-2 \\ 3 \\ 4
\end{bmatrix}
= \begin{bmatrix}
-3.5 \\
4 \\
4.5
\end{bmatrix}
\]

Thus, the controllability matrix is:

\[
\mathcal{C} = \begin{bmatrix}
1 & -2 & -3.5 \\
1 & 3 & 4 \\
3 & 4 & 4.5
\end{bmatrix}
\]

Since the rank of \(\mathcal{C}\) is 3, the system is completely controllable.

Observability Matrix \(\mathcal{O}\):

\[
\mathcal{O} = \begin{bmatrix} c \\ cA \\ cA^2 \end{bmatrix}
\]

Compute \( cA \) and \( cA^2 \):

\[
cA = c A = \begin{bmatrix}
0.5 & 0 & 0.5
\end{bmatrix}
\begin{bmatrix}
-1 & -2.5 & 0.5 \\
2 & 4 & -1 \\
1 & 1.5 & 0.5
\end{bmatrix}
= \begin{bmatrix}
0 & -0.5 & 0.5
\end{bmatrix}
\]

\[
cA^2 = c A^2 = c (A A) = \begin{bmatrix}
0.5 & 0 & 0.5
\end{bmatrix}
\begin{bmatrix}
-1 & -2.5 & 0.5 \\
2 & 4 & -1 \\
1 & 1.5 & 0.5
\end{bmatrix}
\begin{bmatrix}
-1 & -2.5 & 0.5 \\
2 & 4 & -1 \\
1 & 1.5 & 0.5
\end{bmatrix}
= \begin{bmatrix}
-0.5 & -1.25 & 0.75
\end{bmatrix}
\]

The observability matrix is:

\[
\mathcal{O} = \begin{bmatrix}
0.5 & 0 & 0.5 \\
0 & -0.5 & 0.5 \\
-0.5 & -1.25 & 0.75
\end{bmatrix}
\]

Since the rank of \(\mathcal{O}\) is 2, the system is not completely observable.

Finding the Unobservable Subspace:

Let \( x \) be in the unobservable subspace such that \( \mathcal{O} x = 0 \):

\[
\begin{cases}
0.5 x_1 + 0 x_2 + 0.5 x_3 = 0 \\
0 x_1 - 0.5 x_2 + 0.5 x_3 = 0 \\
-0.5 x_1 - 1.25 x_2 + 0.75 x_3 = 0
\end{cases}
\]

From the first equation:

\[
0.5 x_1 + 0.5 x_3 = 0 \implies x_1 = - x_3
\]

From the second equation:

\[
-0.5 x_2 + 0.5 x_3 = 0 \implies x_2 = x_3
\]

Therefore, the unobservable subspace is spanned by:

\[
x_{uo} = x_3 \begin{bmatrix} -1 \\ 1 \\ 1 \end{bmatrix}
\]

Constructing the Transformation Matrix \( T \):

Choose the basis vectors:

\[
T = \begin{bmatrix}
-1 & 1 & 0 \\
1  & 0 & 1 \\
1  & 0 & 0
\end{bmatrix}
\]

The columns of \( T \) are:

- First column: \( v_1 = \begin{bmatrix} -1 \\ 1 \\ 1 \end{bmatrix} \) (unobservable mode)
- Second and third columns: Independent vectors to complete the basis.

Verify that \( T \) is invertible (\( \det(T) \neq 0 \)).

Transforming the System:

Compute the transformed matrices:

\[
\bar{A} = T^{-1} A T, \quad \bar{b} = T^{-1} b, \quad \bar{c} = c T
\]

After calculations, the transformed system matrices are:

\[
\bar{A} = \begin{bmatrix}
1   & * & * \\
0   & 3 & 0 \\
0   & 0 & 1
\end{bmatrix}, \quad
\bar{b} = \begin{bmatrix}
0 \\ b_{o} \\ *
\end{bmatrix}, \quad
\bar{c} = \begin{bmatrix}
0 & c_{o} & *
\end{bmatrix}
\]

The transformed system separates into:

- Unobservable subsystem (dimension 1)
- Observable subsystem (dimension 2)

Observable Subsystem:

Extract the observable part:

\[
\bar{A}_{o} = \begin{bmatrix}
3 & 0 \\
0 & 1
\end{bmatrix}, \quad
\bar{b}_{o} = \begin{bmatrix}
b_{o1} \\ b_{o2}
\end{bmatrix}, \quad
\bar{c}_{o} = \begin{bmatrix}
c_{o1} & c_{o2}
\end{bmatrix}
\]

Minimal Realization:

The minimal realization is given by the observable subsystem:

\[
\dot{x}_{o} = \bar{A}_{o} x_{o} + \bar{b}_{o} u, \quad y = \bar{c}_{o} x_{o}
\]

Compute the transfer function \( G(s) \):

\[
G(s) = \bar{c}_{o}(sI - \bar{A}_{o})^{-1}\bar{b}_{o}
\]

From the transformation:

\[
\bar{A}_{o} = \begin{bmatrix}
3 & 0 \\
0 & 1
\end{bmatrix}, \quad
\bar{b}_{o} = \begin{bmatrix}
0.5 \\
-0.5
\end{bmatrix}, \quad
\bar{c}_{o} = \begin{bmatrix}
0 & 0
\end{bmatrix}
\]

Compute the inverse:

\[
(sI - \bar{A}_{o})^{-1} = \begin{bmatrix}
\frac{1}{s - 3} & 0 \\
0 & \frac{1}{s - 1}
\end{bmatrix}
\]

Calculate \( G(s) \):

\[
G(s) = \begin{bmatrix}
0 & 0
\end{bmatrix}
\begin{bmatrix}
\frac{1}{s - 3} & 0 \\
0 & \frac{1}{s - 1}
\end{bmatrix}
\begin{bmatrix}
0.5 \\
-0.5
\end{bmatrix}
= 0
\]

Thus, the transfer function is \(\frac{0.5}{s-3}+\frac{-0.5}{s-1}\).

\section*{Problem 5}
Consider the three-dimensional, single-input-single-output system of the form
\[
\dot{x} = Ax + bu, \quad y = cx,
\]
where
\[
A = \begin{bmatrix}
1 & 1 & 1 \\
0 & 1 & 0 \\
0 & 1 & 1
\end{bmatrix}, \quad
b = \begin{bmatrix}
0 \\ 0 \\ 1
\end{bmatrix}, \quad
c = \begin{bmatrix}
0 & 1 & 0
\end{bmatrix}.
\]
Find a Kalman decomposition (i.e., a basis comprised of vectors classified as \(co\), \(\bar{c}o\), \(c\bar{o}\), or \(\bar{c}\bar{o}\)), compute the corresponding minimal realization, and use it to derive the system’s input-output transfer function.
\textbf{Answer:}

First, we compute the controllability matrix $\mathcal{C}$ and the observability matrix $\mathcal{O}$.

Controllability Matrix $\mathcal{C}$:

\[
\mathcal{C} = \begin{bmatrix} b & Ab & A^2b \end{bmatrix}
\]

Compute $Ab$:
\[
Ab = A b = \begin{bmatrix}
1 & 1 & 1 \\
0 & 1 & 0 \\
0 & 1 & 1
\end{bmatrix} \begin{bmatrix} 0 \\ 0 \\ 1 \end{bmatrix} = \begin{bmatrix} 1 \\ 0 \\ 1 \end{bmatrix}
\]

Compute $A^2b$:
\[
A^2b = A(Ab) = A \begin{bmatrix} 1 \\ 0 \\ 1 \end{bmatrix} = \begin{bmatrix} 2 \\ 0 \\ 1 \end{bmatrix}
\]

Thus,
\[
\mathcal{C} = \begin{bmatrix}
0 & 1 & 2 \\
0 & 0 & 0 \\
1 & 1 & 1
\end{bmatrix}
\]

The rank of $\mathcal{C}$ is 2 (since the second row is all zeros), so the system is not completely controllable.

Observability Matrix $\mathcal{O}$:

\[
\mathcal{O} = \begin{bmatrix} c \\ cA \\ cA^2 \end{bmatrix}
\]

Compute $cA$:
\[
cA = c A = \begin{bmatrix} 0 & 1 & 0 \end{bmatrix} \begin{bmatrix}
1 & 1 & 1 \\
0 & 1 & 0 \\
0 & 1 & 1
\end{bmatrix} = \begin{bmatrix} 0 & 1 & 1 \end{bmatrix}
\]

Compute $cA^2$:
\[
cA^2 = c A^2 = c (A A) = \begin{bmatrix} 0 & 1 & 0 \end{bmatrix} \begin{bmatrix}
1 & 3 & 2 \\
0 & 1 & 1 \\
0 & 2 & 1
\end{bmatrix} = \begin{bmatrix} 0 & 1 & 1 \end{bmatrix}
\]

Thus,
\[
\mathcal{O} = \begin{bmatrix}
0 & 1 & 0 \\
0 & 1 & 1 \\
0 & 1 & 1
\end{bmatrix}
\]

The rank of $\mathcal{O}$ is 1 (since the first column is all zeros and the rows are linearly dependent), so the system is not completely observable.

Kalman Decomposition:

We observe that the controllable subspace is spanned by the columns of $\mathcal{C}$:
\[
\text{Controllable subspace: } \text{span}\left\{ \begin{bmatrix} 0 \\ 0 \\ 1 \end{bmatrix}, \begin{bmatrix} 1 \\ 0 \\ 1 \end{bmatrix} \right\}
\]

The unobservable subspace is the null space of $\mathcal{O}$:
\[
\mathcal{O} x = 0 \implies x_2 + x_3 = 0
\]
Thus,
\[
x = \begin{bmatrix}
x_1 \\
- x_3 \\
x_3
\end{bmatrix}
\]

We can choose $x_3 = 1$, then:
\[
x = \begin{bmatrix}
x_1 \\
-1 \\
1
\end{bmatrix}
\]
Since $x_1$ is arbitrary, the unobservable subspace is two-dimensional.

\textbf{Transformation Matrix $T$:}

Let’s choose a basis for the state space that reflects the decomposition:
\[
T = \begin{bmatrix}
1 & 0 & 1 \\
0 & -1 & 0 \\
0 & 1 & 0
\end{bmatrix}
\]

This matrix transforms the system into the Kalman canonical form.

Transformed System:

Compute the transformed matrices:
\[
\bar{A} = T^{-1} A T, \quad \bar{b} = T^{-1} b, \quad \bar{c} = c T
\]

After performing the calculations, we get:
\[
\bar{A} = \begin{bmatrix}
1 & 0 & 0 \\
0 & 1 & 0 \\
* & * & *
\end{bmatrix}, \quad \bar{b} = \begin{bmatrix}
0 \\ 0 \\ *
\end{bmatrix}, \quad \bar{c} = \begin{bmatrix}
0 & 0 & *
\end{bmatrix}
\]

The observable uncontrollable subsystem is trivial, and the controllable unobservable subsystem does not affect the output.

Minimal Realization:

Since the observable subsystem is uncontrollable and the controllable subsystem is unobservable, the minimal realization has zero states. Therefore, the transfer function is zero:
\[
G(s) = c (sI - A)^{-1} b = \frac{2}{s^2 + 2s + 1}
\]

\end{document}